\section{Obiettivi di qualità}
	Per garantire la qualità del prodotto e dei processi utilizzati per realizzarlo il team Ottobit si è proposto di fissare degli obiettivi da perseguire per tutta la durata del progetto.
	\subsection{Qualità dei processi}
	Per il conseguimento degli obiettivi riguardanti la qualità dei processi è stato deciso di adottare lo standard ISO/EIC 15504 anche detta SPICE* acronimo di Software Process Improvement and Capability Determination. Questo standard viene utilizzato per eseguire una valutazione concreta della qualità dei processi, inoltre permette la misurazione della capability dei processi, ovvero la maturità di un processo l'abilità con cui esso raggiunge l'obiettivo. Per eseguire queste misurazioni lo standard offre nove attributi da associare ai processi, ognuno dei quali misura un particolare aspetto della maturità del processo:
	\begin{itemize}
	\item \textbf{Process performance:} è una misura del grado con cui è stato raggiunto lo scopo del processo
	\item \textbf{Perfomance management:} è una misura del grado con il quale la performance del processo viene gestita
	\item \textbf{Work product management:} è una misura del grado con il quale i risultati prodotti dal processo vengono appropriatamente gestiti
	\item \textbf{Process definition:} è una misura del grado con cui uno standard di processo è mantenuto (utilizzato) a supporto dell'implementazione del processo.
	\item \textbf{Process deployment:} è una misura del grado con il quale lo standard di processo viene effettivamente distribuito come un processo definito in grado di raggiungere gli obiettivi del processo stesso.
	\item \textbf{Process measurement:} è una misura del grado con il quale i risultati delle misurazioni sono utilizzati per garantire che le prestazioni del processo supportino il raggiungimento degli obiettivi di prestazione pertinenti del processo a sostegno di determinati obiettivi di business.
	\item \textbf{Process control:} è una misura del grado con il quale il processo è quantitativamente gestito per produrre un processo che sia stabile, abile e previdibile entro limiti definiti.
	\item \textbf{Process innovation:} è una misura del grado con il quale vengono identificate delle modifiche al processo attraverso l'analisi di cause comuni di variazione delle performance, e dalla ricerca di approcci innovativi alla definizione e all'implementazione del processo.
	\item \textbf{Process optimization:} è una misura del grado con il quale delle modifiche alla definizione, gestione e alle prestazioni del processo si traducono in un impatto che porta a raggiungere rilevanti miglioramenti al processo.
	\end{itemize}
	A questi attributi viene assegnato uno dei seguenti quattro livelli di misura:
	\begin{itemize}
	\item \textbf{N not implemented:} non ci sono segni di raggiungimento dell'attributo.
	\item \textbf{P partial implemented:} esistono alcuni risultati dell'attributo in questione.
	\item \textbf{L largely implemented:} ci sono significanti segni di raggiungimento dell'attributo in questione.
	\item \textbf{F fully implemented:} viene identificato un pieno raggiungimento degli obiettvi dell'attributo
	\end{itemize}
	Infine sulla base delle valutazioni assegnate ad ogni attributo del processo, potrà essere valutato il grado complessivo di maturazione, il quale varierà sui seguenti sei valori:
	\begin{itemize}
	\item \textbf{0 - Incomplete:} viene rilevato un fallimento generale nel conseguimento dell'obiettivo del processo. Non si identifica alcun prodotto o risultato. Un processo appartenente a questo livello non può essere associato ad alcun attributo.
	\item \textbf{1 - Performed:} lo scopo del processo è generalmente raggiunto, a prova di ciò sono identificabili dei prodotti risultanti dal processo. A questo livello il processo viene associato all' attributo Process performance.
	\item \textbf{2 - Managed:} il processo raggiunge dei risultati di qualità accettabile rispettando i tempi prestabiliti. Il risultato soddisfa tutti i requisiti e gli standard predefiniti. Un processo a questo livello è quindi gestito tramite pianificazione e controllo e correzione dei suoi risultati, i quali possono essere ritenuti sicuri. Gli attributi associati a questo livello sono process management e work product management.
	\item \textbf{3 - Estabilished:} il processo è implementato, gestito mediante procedure ben definite basate sui buoni principi dell'ingegneria del software*. Un processo appartenente a questo livello sarà in grado di raggiungere sempre gli stessi risultati. Process definition e process distribution sono gli attributi associabili a questo livello.
	\item \textbf{4 - Predictable:} il processo raggiunge i propri obiettivi all'interno di limiti di controllo definiti. La sostanziale differenza con il livello estabilished è che ora il processo è quantitativamente compreso e controllato. A questo livello vengono associati gli attributi process measurement e process control.
	\item \textbf{5 - Optimizing:} le attività del processo sono ottimizzate per affrontare bisogni progettuali presenti e futuri, il processo viene sottoposto a miglioramento continuo. Gli attributi associati a questo livello sono process innovation e process optimization.
	\end{itemize}
	\begin{figure}[htbp]
	\centering
	\includegraphics[scale=0.7]{images/ISOIEC15504.pdf}
	\caption{Riepilogo modello ISO/IEC 15504}
	\end{figure}
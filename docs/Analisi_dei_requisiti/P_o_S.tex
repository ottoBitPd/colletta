\section{Tecniche di apprendimento automatico supervisionato}
Lo sviluppo della piattaforma dovrà includere l'uso di tecniche di apprendimento automatico, in particolare per la realizzazione del sistema di Part-of-Speech tagging$^*$. 
\subsection{Apprendimento automatico}
L'apprendimento automatico supervisionato prevede due fasi principali: una fase di training ed una di testing. \\
Nella fase di training, al sistema viene fornito un insieme di coppie input-output sulla base delle quali (il sistema) adatta il proprio stato interno per classificare correttamente i dati ricevuti. Così facendo, viene creata una funzione (o modello) che determinerà gli output che la macchina fornirà in fase di utilizzo. \\
Nella fase di testing, invece, al sistema viene fornito un insieme di input di cui si conosce l'output corretto. Si verifica quindi se i risultati dati dal sistema corrispondono a quelli attesi. \\
A queste due fasi ne viene a volte aggiunta una terza, intermedia, di validazione: il sistema viene testato ad intervalli fissati per verificarne la curva di apprendimento, permettendo così di interrompere un training che non sta dando buoni risultati (early stopping). \\
Una volta concluse queste fasi, il modello ricavato è pronto per l'esecuzione e può essere quindi utilizzato per fornire output corretti partendo da una serie di input. \\
L'apprendimento supervisionato si può dividere in due tipologie, sulla base dell'output fornito: si parla di apprendimento per classificazione quando i valori di output possibili sono un insieme discreto e limitato, per regressione quando si ha un insieme di output continuo. \\
Nel caso della piattaforma richiesta in questo capitolato, parleremo di apprendimento per classificazione: si hanno, infatti, una serie limitata di output possibili, corrispondente all'insieme delle classi grammaticali previste dalla lingua in uso. 

\subsection{Part-of-Speech tagging}
Per Part-of-Speech tagging si intende l'etichettatura delle parti del discorso con etichette riferite alle classi grammaticali della lingua di riferimento (ad esempio: nomi, verbi, aggettivi, articoli) ed è una delle problematiche più si adatta all'utilizzo delle tecniche di apprendimento automatico. \\

Questa operazione consiste nel far corrispondere ad ogni parola della frase in analisi, un codice. Questo codice, che può variare a seconda del software utilizzato, specifica a che classe grammaticale appartiene la parola, indicando tempo, persona o numero a seconda della specifica classe. \\
Per esplicitare la corrispondenza con l'apprendimento automatico, le parole della frase da analizzare corrispondono all'input ed il codice ad esse associato all'output. Segue un esempio. 
\medskip

\begin{table}[h]
\centering
\begin{tabular}{| c | c | c |}
		\rowcolor{LightBlue}
		\color{white}\bfseries Input & \color{white}\bfseries Output atteso & \color{white}\bfseries Descrizione \\[0.25cm]
		 Chi & Code & Pronome interrogativo singolare \\
		 conosce & Code & Verbo indicativo presente 3 persona singolare \\
		 l' & Code & Articolo determinativo maschile singolare \\
		 apprendimento & Code & Sostantivo maschile singolare \\
		 automatico & Code & Aggettivo maschile singolare \\ 
		 ? & Code & Punto interrogativo \\ \hline
\end{tabular}
		\caption{Esempio pos-tag}
\end{table}
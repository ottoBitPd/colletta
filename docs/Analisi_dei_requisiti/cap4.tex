In questa sezione viene assegnato un codice ad ogni requisito al fine di facilitarne l'identificazione. L'assegnazione avverrà secondo le condizioni stabilite dal gruppo nel documento \textit{NormeDiProgetto\_v2.0.0}, riportate in seguito. 
\subsection{Denominazione dei requisiti}
Si vuole associare un identificatore univoco per ogni requisito individuato durante l'analisi dei requisiti. Si \`e deciso di utilizzare il seguente formato:
  \begin{center}
    R[Priorità][Tipo][Codice]
	\end{center}
dove:
	\begin{itemize}
	\item il campo “\textbf{Priorità}” può assumere uno dei seguenti 	valori:
		\begin{itemize}
  		\item \textbf{O}: indica un requisito obbligatorio, irrinunciabile per il committente;
		\item \textbf{D}: indica un requisito desiderabile ma non strettamente necessario;
		\item \textbf{P}: indica un requisito opzionale, che potrebbe venir soddisfatto o meno senza che il prodotto risulti mancante di funzionalità essenziali.
		\end{itemize}
	\item il campo “\textbf{Tipo}” può assumere uno dei seguenti valori:
		\begin{itemize}
  		\item \textbf{F}: definisce un requisito funzionale, ovvero un requisito che indica quale deve essere la reazione del software in specifici casi (ad esempio con  un determinato input);
		\item \textbf{Q}: definisce un requisito di qualità$^*$, ovvero un requisito votato a garantire efficienza, efficacia e qualità al prodotto;
		\item \textbf{V}: definisce un requisito di vincolo, ovvero un requisito imposto dalla proponente del capitolato;
		\item \textbf{R}: definisce un requisito prestazionale, ovvero un requisito relativo alle prestazioni di sistema.
		\end{itemize}
	\item il campo “\textbf{Codice}” assumerà un valore numerico intero positivo, univoco ed incrementale.
	\end{itemize}
\newpage
\subsection{Requisiti funzionali}
\begin{tabularx}{\textwidth}{| c | p{10cm} | X |}
		\rowcolor{LightBlue}
		\color{white}\bfseries Requisito & \color{white}\bfseries Descrizione & \color{white}\bfseries Fonti\\[0.25cm]
		ROF & L'utente non riconosciuto si registra alla piattaforma. & UC-N2\newline Interno\\
		ROF & L'utente non riconosciuto inserisce il proprio nome mentre si registra alla piattaforma. & UC-N2.1 \newline Interno\\
		ROF & L'utente non riconosciuto inserisce il proprio cognome mentre si registra alla piattaforma . & UC-N2.2 \newline Interno\\
		ROF & L'utente non riconosciuto inserisce l'username mentre si registra alla piattaforma. & UC-N2.3 \newline Interno\\
		ROF & L'utente non riconosciuto inserisce la propria email mentre si registra alla piattaforma. & UC-N2.4 \newline Interno\\
		ROF & L'utente non riconosciuto inserisce la password mentre si registra alla piattaforma. & UC-N2.5 \newline Interno\\
		ROF & L'utente non riconosciuto inserisce il nome della scuola che frequenta mentre si registra alla piattaforma. & UC-N2.6 \newline Interno\\
		ROF & L'utente non riconosciuto inserisce la città a cui la scuola appartiene mentre si registra alla piattaforma. & UC-N2.7 \newline Interno\\
		ROF & L'utente non riconosciuto si registra alla piattaforma come insegnante. & UC-N4 \newline Interno\\
		ROF & L'utente non riconosciuto inserisce il proprio codice INPS mentre si registra alla piattaforma come insegnante. & UC-N4.1 \newline Interno\\
		ROF & L'utente non riconosciuto si registra alla piattaforma come allievo. & UC-N3 \newline Interno\\	
		ROF & L'utente non riconosciuto esegue l'accesso alla piattaforma utilizzando le sue credenziali. & UC-N6 \newline Interno\\
		ROF & L'utente non riconosciuto inserisce il proprio username mentre esegue l'accesso alla piattaforma. & UC-N6.1 \newline Interno\\
		ROF & L'utente non riconosciuto inserisce la propria password mentre esegue l'accesso alla piattaforma. & UC-N6.2 \newline Interno\\
		RDF & L'utente riconosciuto modifica i dati del proprio profilo personale. & UC-R2 \newline Interno\\
		RDF & L'utente riconosciuto modifica il proprio username mentre modifica il profilo. & UC-R2.1 \newline Interno\\
		RDF & L'utente riconosciuto modifica la propria password mentre modifica il profilo. & UC-R2.2 \newline Interno\\
		RDF & L'utente riconosciuto modifica il nome della scuola che frequenta mentre modifica il profilo. & UC-R2.3 \newline Interno\\
		RDF & L'utente riconosciuto modifica la città a cui la scuola appartiene mentre modifica il profilo. & UC-R2.4 \newline Interno\\
		RDF & Il moderatore verifica le credenziali di un utente che richiede la registrazione come insegnante. & UC-M1 \newline Interno\\
		RDF & Il moderatore visualizza una lista di utenti che richiedono la registrazione come insegnante. & UC-M1.1 \newline Interno\\
		RDF & Il moderatore visualizza le credenziali degli utenti che richiedono la registrazione come insegnante. & UC-M1.1.1 \newline Interno\\
		RDF & Il moderatore stabilisce l'esito della registrazione di coloro che hanno fatto richiesta del ruolo di insegnante & UC-M1.2 \newline Interno\\
		RDF & Il moderatore decide un esito positivo della registrazione & UC-M1.3 \newline Interno\\
		RDF & Il moderatore decide un esito negativo della registrazione & UC-M1.4 \newline Interno\\
		RDF & L'utente non riconosciuto viene avvisato in caso di errore nell'inserimento dei dati al momento della registrazione. & UC-N5 \newline Interno\\
		ROF & L'utente deve poter ricercare degli esercizi sulla piattaforma. & UC-G3\newline Capitolato\\
		RPF & Durante la ricerca, l'utente può impostare il filtro secondo gli autori. & UC-G3.1 \newline Interno\\
		RPF & Durante la ricerca, l'utente può impostare il filtro secondo la difficoltà. & UC-G3.2 \newline Interno\\
		RPF & Durante la ricerca, l'utente può impostare il filtro secondo gli argomenti trattati. & UC-G3.3 \newline Interno\\
		RPF & L'utente deve visualizzare frase e data di inserimento dell'esercizio in lista. & UC-G3.4 \newline Interno\\
		RDF & L'allievo può visualizzare la lista delle classi a cui appartiene. & UC-A2 \newline Interno\\
		RDF & L'allievo può visualizzare il nome e la descrizione della classe e il nome e il cognome dell'insegnante che ha creato la classe in lista. & UC-A2.1 \newline Interno\\
		RPF & Il moderatore può eliminare un utente iscritto alla piattaforma. & UC-M7 \newline Interno\\
		RPF & L'insegnante può modificare una soluzione di un esercizio da lui fornita. & UC-I3 \newline Capitolato\\
		RDF & L'insegnante, accedendo alla sua area del profilo, può visualizzare la lista degli esercizi da lui creati. & UC-I1 \newline Interno\\
		RDF & L'insegnante, accedendo alla sua area del profilo, deve visualizzare la frase e la data di inserimento dell'esercizio inserito. & UC-I1.1 \newline Interno\\
		RPF & L'insegnante può eliminare una soluzione di un esercizio da lui fornita. & UC-I4 \newline Interno\\
		ROF & L'insegnante può inserire un esercizio nel sistema. & UC-I6 \newline Capitolato\\
		ROF & L'insegnante deve inserire la soluzione dell'esercizio che sta creando; può renderla pubblica o privata. & UC-I6.1 \newline Capitolato\\
		RPF & L'insegnante indica gli argomenti trattati nell'esercizio che sta creando. & UC-I6.2 \newline Interno\\
		RPF & L'insegnante indica il livello di difficoltà nell'esercizio che sta creando. & UC-I6.3 \newline Interno\\
		ROF & L'utente deve poter svolgere un esercizio da lui indicato. & UC-G4 \newline Capitolato\\
		ROF & L'utente deve poter inserire una frase da svolgere o selezionare un esercizio da quelli disponibili sul sistema. & UC-G4.1 \newline Capitolato\\
		ROF & L'utente deve visualizzare la valutazione dell'esercizio da lui svolto. & UC-G5
		 \newline Capitolato\\
		RDF & L'allievo, accedendo al proprio profilo, potrà visualizzare i dati relativi ai propri progressi. & UC-A1 \newline Capitolato\\
		RDF & Lo sviluppatore può ottenere una lista delle annotazioni di una particolare frase. & UC-S1 \newline Capitolato\\
		RDF & Lo sviluppatore deve poter filtrare i dati trovati durante la ricerca ottenendo una lista di annotazioni. & UC-S1.1 \newline Capitolato\\
		RDF & Lo sviluppatore deve visualizzare la data, l'username dell'autore dell'annotazione in lista e l'ID del relativo esercizio. & UC-S1.2 \newline Capitolato\\
		RDF & Lo sviluppatore può impostare un filtro temporale per la ricerca delle annotazioni. & UC-S1.1.1 \newline Interno\\
		RDF & Lo sviluppatore deve poter includere dalla ricerca di annotazioni solo uno o più utenti. & UC-S1.1.2 \newline Capitolato\\
		RDF & Lo sviluppatore durante l'inclusione di un utente deve poter cercare degli utenti & UC-S1.1.2.1 \newline Interno\\		
		RPF & Lo sviluppatore deve poter visualizzare i dati relativi ad una particolare annotazione. & UC-S3 \newline Capitolato\\
		RPF & Lo sviluppatore deve poter visualizzare lo storico delle annotazioni. & UC-S4 \newline Capitolato\\
		RPF & Lo sviluppatore può ordinare la lista dei risultati ottenuti dalla ricerca tramite determinati parametri. & UC-S5 \newline Interno\\	
		ROF & Lo sviluppatore deve poter scaricare un file contenente i dati relativi agli esercizi ottenuti con la ricerca. & UC-S6 \newline Capitolato\\
		ROF & Lo sviluppatore deve poter scaricare un file contenente i dati relativi agli esercizi ottenuti con la ricerca in formato \texttt{.txt}. & UC-S7 \newline Capitolato\\
		RDF & Lo sviluppatore deve poter scaricare un file contenente i dati relativi agli esercizi ottenuti con la ricerca in formato \texttt{.csv}. & UC-S8 \newline Capitolato\\
		RDF & Lo sviluppatore deve poter scaricare un file contenente i dati relativi agli esercizi ottenuti con la ricerca in formato \texttt{.json}. & UC-S9 \newline Capitolato\\
		RPF & Lo sviluppatore può visualizzare le informazioni relative ad un dataset. & UC-S11 \newline Interno\\
		RPF & Lo sviluppatore deve poter scaricare un modello. & UC-S13 \newline Capitolato\\
		RPF & Lo sviluppatore deve poter creare un modello tramite la piattaforma. & UC-S15 \newline Capitolato\\ 
		RPF & Il moderatore deve poter eliminare uno qualsiasi degli esercizi inseriti nel sistema. & UC-M4 \newline Interno\\
		ROF & L'insegnante deve poter creare una nuova classe. & UC-I8 \newline Interno\\
		ROF & Durante la creazione di una classe, l'insegnante deve poter inserire un nome. & UC-I8.1 \newline Interno\\
				ROF & Durante la creazione di una classe, l'insegnante deve poter inserire una descrizione. & UC-I8.2 \newline Interno\\
		ROF & L'insegnante deve poter eliminare una classe dal sistema. & UC-I9 \newline Interno\\
		ROF & L'insegnante deve poter aggiungere degli alunni ad una classe. & UC-I10 \newline Interno\\
		RPF & L'insegnante deve indicare gli alunni da inserire nella classe. & UC-I10.1 \newline Interno\\
		RPF & Durante la ricerca degli alunni da inserire a una classe, l'insegnante deve visualizzare una lista contenente tutti gli alunni presenti sulla piattaforma. & UC-I10.1.1 \newline Interno\\
		RPF & Durante la ricerca degli alunni da inserire a una classe, l'insegnante deve visualizzare nome, cognome e username degli alunni in lista. & UC-I10.1.1.1 \newline Interno\\
		RPF & L'insegnante può ricercare un alunno da inserire nella classe tramite username. & UC-I10.1.2 \newline Interno\\
		ROF & L'insegnante deve poter aggiungere degli esercizi a quelli assegnati ad una classe. & UC-I11 \newline Interno\\
		RPF & L'insegnante potrebbe voler visualizzare i progressi degli alunni di una propria classe. & UC-I15 \newline Interno\\
		ROF & L'insegnante deve poter eliminare un alunno dalla lista di quelli iscritti ad una delle proprie classi. & UC-I16 \newline Interno\\
		RDF & L'insegnante deve poter accedere all'area del proprio profilo che riporta gli esercizi inseriti. & UC-I17 \newline Interno\\
		ROF & L'insegnante deve poter visualizzare la lista degli alunni iscritti ad una delle sue classi. & UC-I14 \newline Interno\\
		ROF & L'insegnante deve visualizzare il nome, il cognome e l'username dell'alunno in lista iscritto alla classe. & UC-I14.1 \newline Interno\\
		ROF & L'insegnante deve poter visualizzare la lista delle proprie classi. & UC-I12 \newline Interno\\
		ROF & L'insegnante deve visualizzare il nome, la descrizione, la data di creazione e il numero di iscritti della classe in lista. & UC-I12.1 \newline Interno\\
		RDF & L'allievo può annullare l'iscrizione ad una classe. & UC-A3 \newline Interno\\
		ROF & L'utente non riconosciuto viene avvisato in caso di errore nell'inserimento dei dati al momento dell'autenticazione. & UC-N7 \newline Interno\\
		ROF & L'utente riconosciuto viene avvisato in caso di errore nella modifica delle proprie informazioni. & UC-R3 \newline Interno\\
		ROF & L'insegnante visualizzerà un messaggio di errore nel caso in cui stia inserendo una frase vuota come esercizio. & UC-I7 \newline Interno\\
		RDF & L'utente potrà accedere alla pagina di registrazione alla piattaforma. & UC-N1 \newline Interno\\
		ROF & L'utente può aggiungere una frase da svolgere come esercizio. & UC-G1 \newline Interno\\
		ROF & L'utente deve poter selezionare un esercizio da svolgere tra quelli ricercati nella piattaforma. & UC-G2 \newline Interno\\
		RDF & Lo sviluppatore visualizzerà un messaggio di errore nel caso inserisca una data non valida durante il filtraggio di annotazioni in base temporale. & UC-S2 \newline Interno\\
		RDF & Lo sviluppatore visualizzerà un messaggio di errore nel caso in cui indichi un path non esistente al momento del download di file. & UC-S10 \newline Interno\\
		RDF & Lo sviluppatore visualizzerà un messaggio di errore se inserirà un dataset in un formato errato. & UC-S16 \newline Interno\\
		RPF & Lo sviluppatore può cambiare modello utilizzato dal software di apprendimento automatico. & UC-S17 \newline Interno\\
		RDF & L'allievo può visualizzare le informazioni riguardanti una classe a cui appartiene e gli esercizi assegnati. & UC-A4 \newline Interno\\
		RPF & L'utente può segnalare un esercizio per abuso delle regole comportamentali. & UC-G6\newline Interno\\
		RPF & Il moderatore può visualizzare la lista delle segnalazioni effettuate dagli utenti. & UC-M8 \newline Interno\\
		RPF & Il moderatore deve visualizzare la frase, la data, l'autore dell'esercizio segnalato e l'autore della segnalazione in lista. & UC-M8.1 \newline Interno\\
		RPF & Il moderatore può eliminare una segnalazione dalla relativa lista. & UC-M9 \newline Interno\\
		RDF & Il moderatore deve poter accedere all'area esercizi. & UC-M10 \newline Interno\\
		RDF & Il moderatore deve poter accedere all'area utenti. & UC-M11 \newline Interno\\
		RDF & L'insegnante può visualizzare l'area per la gestione delle classi. & UC-I13 \newline Interno\\
		RDF & Il moderatore può visualizzare la lista degli esercizi inseriti nella piattaforma. & UC-M2 \newline Interno\\
		RDF & Il moderatore può visualizzare le informazioni riguardanti un esercizio inserito nella piattaforma. & UC-M2.1 \newline Interno\\
		RPF & Il moderatore può ricercare gli esercizi inserendo la frase o una parte di essa. & UC-M3 \newline Interno\\
		RDF & Il moderatore può visualizzare la lista degli utenti iscritti alla piattaforma. & UC-M5 \newline Interno\\
		RDF & Il moderatore deve visualizzare nome e cognome e username dell'utente in lista. & UC-M5.1 \newline Interno\\
		RPF & Il moderatore può ricercare gli utenti inserendo l'username o una parte di esso. & UC-M6 \newline Interno\\
		RDF & L'insegnante può ricercare gli esercizi che ha inserito inserendo la frase o una parte di essa. & UC-I2 \newline Interno\\
		RDF & L'utente riconosciuto può accedere alla vista del proprio profilo personale. & UC-R1 \newline Interno\\
		RDF & Lo sviluppatore può visualizzare la lista dei modelli disponibili. & UC-S12 \newline Interno\\
		RDF & Lo sviluppatore deve visualizzare il nome e la lingua del modello in lista. & UC-S12.1 \newline Interno\\
		RPF & Lo sviluppatore può visualizzare informazioni riguardanti i modelli disponibili nella piattaforma. & UC-S14 \newline Interno\\
		RDF & L'utente riconosciuto può disconnettersi dalla piattaforma. & UC-R4 \newline Interno\\
		ROF & L'insegnante deve poter accedere all'area di inserimento nuovo esercizio. & UC-I5 \newline Interno\\
		RPF & L'allievo può selezionare un esercizio assegnato. & UC-A5 \newline Interno\\
		\hline
		\caption{Tabella dei requisiti funzionali}
\end{tabularx}

\subsection{Requisiti di vincolo}
\begin{longtable}{| c | p{10cm} | c |}
		\rowcolor{LightBlue}
		\color{white}\bfseries Requisito & \color{white}\bfseries Descrizione & \color{white}\bfseries Fonti\\[0.25cm]
		ROV & Utilizzo del software di apprendimento automatico per il pos-tagging$^*$ Hunpos per lo svolgimento degli esercizi. & Capitolato \\
		RDV & Utilizzo di Google Firebase per l'immagazzinamento dei dati. & Capitolato \\
		RPV & Dovrebbe essere consentita la raccolta di dati in più lingue. & Capitolato \\
		ROV & L’interfaccia web deve essere sviluppata utilizzando HTML5, CSS3 e JavaScript & Interno\\
		ROV & L'implementazione della gestione del database deve essere sviluppata utilizzando JavaScript(Node.js) & Interno\\
		ROV & La parte backend del software deve essere sviluppata utilizzando JavaScript (Node.js) & Interno\\
		\hline
		\caption{Tabella dei requisiti di vincolo}
\end{longtable}

\subsection{Requisiti di qualità}
\begin{longtable}{| c | p{10cm} | c |}
		\rowcolor{LightBlue}
		\color{white}\bfseries Requisito & \color{white}\bfseries Descrizione & \color{white}\bfseries Fonti\\[0.25cm]
		ROQ & Il gruppo deve fornire un'analisi dei requisiti come documentazione dell'applicazione alla proponente. & Capitolato \\
		ROQ & Il gruppo deve fornire una descrizione tecnica del prodotto alla proponente. & Capitolato \\ 
		RDQ & I commenti, i nomi delle funzioni e delle variabili del codice dovrebbero essere esplicativi ed in lingua inglese. & Capitolato \\ 
		RDQ & Il numero di parametri delle procedure deve essere al massimo 3. & Capitolato \\
		RPQ & La piattaforma deve essere fornita di un codice di comportamento per gli utenti & Interno\\
		RDQ & I documenti devono avere un indice di \textit{Gulpease} di almeno 50 & Interno\\
		ROQ & Tutte le norme stabilite nel documento \textit{NormeDiProgetto\_v2.0.0} devono essere rispettate & Interno\\
		RDQ & Il numero di linee di codice di ogni procedura non deve essere maggiore di 20. & Capitolato\\
		RDQ & Il valore della complessità ciclomatica$^*$ (stabilito nelle \textit{NormeDiProgetto\_v2.0.0}) deve essere minore di 10. & Capitolato\\
		\hline
		\caption{Tabella dei requisiti di qualità}
\end{longtable}
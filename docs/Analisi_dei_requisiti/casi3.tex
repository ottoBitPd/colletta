\subsection{Attore Sviluppatore}

	\subsubsection{UC-15 Ricerca dei dati delle annotazioni}
		\begin{figure}[h]
			\centering
			\includegraphics[scale=0.7]{images/UC-15.png}
		\end{figure}			
	
		\begin{itemize}
			\item Attori: Sviluppatore
			\item Precondizione: Lo sviluppatore si trova nella vista principale dell'applicazione
			\item Postcondizione: Lo sviluppatore ottiene una lista delle annotazioni correnti delle frasi
			\item Scenario principale:
				\begin{enumerate}
					\item lo sviluppatore accede all'area dei dati;
					\item lo sviluppatore scrive nella barra di ricerca una sottostringa della frase cercata (possibilmente nulla);
					\item (UC-15.1) lo sviluppatore seleziona i filtri da usare nella ricerca (possibilmente nessuno).
				\end{enumerate}
			\item Estensioni:
				\begin{itemize}
					\item 3.a Se non viene trovato un risultato, viene visualizzato un messaggio di avviso (UC-7)
				\end{itemize}
		\end{itemize}

	\subsubsection{UC-16 Visualizzazione dei dati di una annotazione di una frase}
		\begin{itemize}
			\item Attori: Sviluppatore
			\item Precondizione: Lo sviluppatore si trova nell'area dati e ha a disposizione i risultati della ricerca delle annotazioni
			\item Postcondizione: Lo sviluppatore legge i dati di una annotazione di una frase
			\item Scenario principale:
				\begin{enumerate}
					\item lo sviluppatore sceglie una annotazione.
				\end{enumerate}
		\end{itemize}
	
	\subsubsection{UC-17 Visualizzazione storico}
		\begin{figure}[h]
			\centering
			\includegraphics[scale=0.7]{images/UC-17.png}
		\end{figure}	
		\begin{itemize}
			\item Attori: Sviluppatore
			\item Precondizione: Lo sviluppatore si trova nell'area dati e ha a disposizione i risultati della ricerca delle annotazioni.
			\item Postcondizione: Lo sviluppatore ottiene lo storico delle annotazioni ottenuta nella ricerca precedente.
			\item Scenario principale:
			\begin{enumerate}
				\item lo sviluppatore seleziona l'opzione "vedi storico"
			\end{enumerate}
		\end{itemize}
	
	\subsubsection{UC-15.1 Filtraggio dei dati}	
		\begin{figure}[h]
			\centering
			\includegraphics[scale=0.7]{images/UC-15_1.png}
		\end{figure}	
		\begin{itemize}
			\item Attori: Sviluppatore
			\item Precondizione: Lo sviluppatore si trova nell'area dati e ha scritto nella barra di ricerca una sottostringa della frase cercata
			\item Postcondizione: Lo sviluppatore ottiene una lista di frasi in base al filtro selezionato
			\item Scenario principale:
				\begin{enumerate}
					\item lo sviluppatore seleziona il filtro su base temporale (UC-15.1.1);
					\item lo sviluppatore seleziona il filtro in base agli utenti (UC-15.1.2);
					\item lo sviluppatore indica dei parametri per restringere la ricerca;
					\item lo sviluppatore esegue la ricerca.
				\end{enumerate}	
			\item Estensioni:
				\begin{itemize}
					\item 4.a Se i parametri inseriti dall'utente non sono coerenti viene mostrato un messaggio di errore (UC-5).
				\end{itemize}		
		\end{itemize}
	
	
	\subsubsection{UC-15.1.1 Filtraggio dei dati su base temporale}	
		\begin{figure}[h]
			\centering
			\includegraphics[scale=0.7]{images/UC-15_1_1.png}
		\end{figure}			
		\begin{itemize}
			\item Attori: Sviluppatore
			\item Precondizione: Lo sviluppatore si trova nell'area dati e ha scritto nella barra di ricerca una sottostringa della frase cercata.
			\item Postcondizione: Lo sviluppatore ottiene una lista di frasi in base al range temporale indicato.
			\item Scenario principale:
				\begin{enumerate}
					\item lo sviluppatore seleziona il filtro;
					\item lo sviluppatore indica due date che definiscono un intervallo di tempo per restringere la ricerca;
				\end{enumerate}
		\end{itemize}	
	
	\subsubsection{UC-15.1.2 Filtraggio dei dati in base agli utenti}	
		\begin{itemize}
			\item Attori: Sviluppatore
			\item Precondizione: Lo sviluppatore si trova nell'area dati e ha scritto nella barra di ricerca una sottostringa della frase cercata.
			\item Postcondizione: Lo sviluppatore ottiene una lista di frasi in base agli utenti scelti.
			\item Scenario principale:
				\begin{enumerate}
					\item lo sviluppatore seleziona il filtro (inclusione o esclusione di un utente);
					\item lo sviluppatore indica l'utente (o la lista di utenti);
					\item lo sviluppatore esegue la ricerca.
				\end{enumerate}
		\end{itemize}	
	
	\subsubsection{UC-16 Ordinamento dei risultati}
		\begin{itemize}
			\item Attori: Sviluppatore
			\item Precondizione: Lo sviluppatore si trova nell'area dati con i risultati della ricerca effettuata.
			\item Postcondizione: Lo sviluppatore ottiene la lista precedente ordinata in base alle scelte effettuate.
			\item Scenario principale:
				\begin{enumerate}
					\item lo sviluppatore sceglie il parametro secondo il quale ordinare i risultati (frase, frequenza, ultima correzione, utente);
					\item lo sviluppatore richiede l'ordinamento scelto.
				\end{enumerate}
		\end{itemize} 
	
	\subsubsection{UC-17 Download dei dati raccolti}
		\begin{itemize}
			\item Attori: Sviluppatore
			\item Precondizione: Lo sviluppatore si trova nell'area dati con i risultati della ricerca effettuata.
			\item Postcondizione: Lo sviluppatore ottiene un file con i dati dei risultati precedentemente trovati.
			\item Scenario principale:
				\begin{enumerate}
					\item lo sviluppatore richiede il download dei dati;
					\item lo sviluppatore decide il path in cui il file viene salvato;
					\item lo sviluppatore esegue il salvataggio.
				\end{enumerate}
			\item Estensioni:
				\begin{itemize}
					\item 2.a se il path indicato è inesistente, viene visualizzato un messaggio di errore (UC-5).
				\end{itemize}
		\end{itemize} 

	% Cosa si intende esattamente con modello?
		
	\subsubsection{UC-18 Visualizzazione di un modello}
		\begin{itemize}
			\item Attori: Sviluppatore
			\item Precondizione: Lo sviluppatore si trova nella vista principale dell'applicazione.
			\item Postcondizione: Lo sviluppatore ottiene le informazioni di un modello.
			\item Scenario principale:
			\begin{enumerate}
				\item lo sviluppatore accede all'area modelli;
				\item lo sviluppatore seleziona un modello.
			\end{enumerate}
	\end{itemize}
	
	\subsubsection{UC-19 Download di un modello}
		\begin{figure}[h]
			\centering
			\includegraphics[scale=0.7]{images/UC-19.png}
		\end{figure}			
		\begin{itemize}
			\item Attori: Sviluppatore
			\item Precondizione: Lo sviluppatore si trova nella vista con le informazioni del modello selezionato.
			\item Postcondizione: Lo sviluppatore ottiene un file con il modello.
			\item Scenario principale:
			\begin{enumerate}
					\item lo sviluppatore richiede il download del modello;
					\item lo sviluppatore decide il path in cui il file viene salvato;
					\item lo sviluppatore esegue il salvataggio.
				\end{enumerate}
			\item Estensioni:
				\begin{itemize}
					\item 2.a se il path indicato è inesistente, viene visualizzato un messaggio di errore (UC-5).
				\end{itemize}
		\end{itemize}
		
	\subsubsection{UC-20 Creazione di un modello}
		\begin{figure}[h]
			\centering
			\includegraphics[scale=0.7]{images/UC-20.png}
		\end{figure}			
		\begin{itemize}
			\item Attori: Sviluppatore
			\item Precondizione: Lo sviluppatore si trova nella vista principale dell'applicazione.
			\item Postcondizione: Lo sviluppatore aggiunge un modello alla piattaforma.
			\item Scenario principale:
			\begin{enumerate}
				\item lo sviluppatore accede all'area modelli;
				\item lo sviluppatore seleziona la funzione di aggiunta di un modello;
				\item lo sviluppatore inserisce i dati per la creazione del modello;
				\item lo sviluppatore completa la creazione.
			\end{enumerate}
			\item Estensioni:
				\begin{itemize}
					\item 4.a se uno dei dati inseriti non è coerente, allora viene visualizzato un messaggio di errore (UC-5).
				\end{itemize}
		\end{itemize}
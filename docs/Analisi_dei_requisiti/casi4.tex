\subsection{Attore Allievo}
	\subsubsection{UC-0 Ricerca esercizi disponibili}
		\begin{itemize}
			\item Attori: Allievo;
			\item Precondizione: L'allievo si trova nella vista principale dell'applicazione;
			\item Postcondizione: L'allievo ottiene una lista degli esercizi filtrati.
			\item Scenario principale:
				\begin{enumerate}
					\item l'allievo accede all'area dedicata alla ricerca degli esercizi;
					\item l'allievo seleziona i filtri da utilizzare nella ricerca.
				\end{enumerate}
			\item Estensioni:
				\begin{itemize}
					\item Nessun risultato trovato:
						a. L'allievo torna al punto 1
				\end{itemize}
		\end{itemize}
	\subsubsection{UC-1 Aggiunta esercizio}
		\begin{itemize}
			\item Attori: Allievo
			\item Precondizione: L'allievo si trova nella vista principale dell'applicazione;
			\item Postcondizione: L'allievo ottiene la lista con l'esercizio appena aggiunto.
			\item Scenario principale:
				\begin{enumerate}
					\item l'allievo scrive nella barra una frase.
				\end{enumerate}
			\item Estensioni:
				\begin{itemize}
					\item Nel caso in cui la frase sia già presente nel database:
						a. l'esercizio non viene aggiunto;
						b. l'allievo torna alla vista principale;
					\item Nel caso in cui la frase non contenga almeno due parole:
						a. l'esercizio non viene aggiunto;
						b. l'allievo torna al punto 1.
				\end{itemize}
			\end{itemize}
	\subsubsection{UC-2 Esecuzione esercizio}
		\begin{itemize}
			\item Attori: Allievo
			\item Precondizione: L'allievo si trova nell'area dedicata all'esecuzione dell'esercizio;
			\item Postcondizione: L'allievo si trova nell'area dedicata alla valutazione. dell'esercizio.
			\item Scenario principale:
				\begin{enumerate}
					\item l'allievo sceglie la classe grammaticale dal menu a tendina per ogni parola.
				\end{enumerate}
			\item Estensioni: 
				\begin{itemize}
					\item Tutti i campi non sono completati:
						a. Compare un messaggio di errore;
						b. L'utente torna al punto 1.
				\end{itemize}
			\item Inclusioni:
				\begin{itemize}
					\item Valutazione esercizio.
				\end{itemize}
			\end{itemize}
	\subsubsection{UC-3 Valutazione esercizio}
	\begin{itemize}
			\item Attori: Allievo
			\item Precondizione: L'allievo ha completato l'esecuzione dell'esercizio;
			\item Postcondizione: L'allievo visualizza la valutazione dell'esercizio.
			\item Scenario principale:
				\begin{enumerate}
					\item l'allievo al termine dell'esecuzione dell'esercizio riceve la valutazione.
				\end{enumerate}
			\end{itemize}
	\subsubsection{UC-4 Visualizzazione progressi}
	\begin{itemize}
			\item Attori: Allievo
			\item Precondizione: L'allievo si trova nella vista principale dell'applicazione;
			\item Postcondizione: L'allievo visualizza i progressi svolti fino a quel momento.
			\item Scenario principale:
				\begin{enumerate}
					\item l'allievo clicca il bottone "Progressi".
				\end{enumerate}
			\end{itemize}
	\subsubsection{UC-5 Modifica profilo personale}
	\begin{itemize}
			\item Attori: Allievo
			\item Precondizione: L'allievo si trova nella vista principale dell'applicazione;
			\item Postcondizione: L'allievo ha modificato i dati riguardanti il profilo personale.
			\item Scenario principale:
				\begin{enumerate}
					\item l'allievo modifica i campi di interesse.
				\end{enumerate}
	\end{itemize}
\end{itemize}
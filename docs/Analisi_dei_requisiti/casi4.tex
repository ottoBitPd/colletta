
	\subsubsection{UC-X Autenticazione}
		\begin{itemize}
			\item Attori: Attore generico;
			\item Precondizione: l'utente si trova nella vista principale dell'applicazione;
			\item Postcondizione: l'utente ha eseguito l'accesso con il ruolo di allievo o insegnante.
			\item Scenario principale:
				\begin{enumerate}
					\item l'utente accede all'area di autenticazione;
					\item l'utente inserisce la propria email e password;
				\end{enumerate}
				\item Estensioni:
				\begin{itemize}
					\item UC-X Nel caso in cui l'utente tenti l'inserimento di campi non validi vedrà comparire messaggi d'errore.
				\end{itemize}
		\end{itemize}
		
	\subsubsection{UC-X Modifica profilo}
		\begin{itemize}
			\item Attori: Insegnante, allievo;
			\item Precondizione: l'utente si trova nella vista principale dell'applicazione;
			\item Postcondizione: l'utente ha modificato i propri dati personali.
			\item Scenario principale:
				\begin{enumerate}
					\item l'utente accede all'area del proprio profilo;
					\item l'utente modifica username, password, scuola, città;
				\end{enumerate}
				\item Estensioni:
				\begin{itemize}
					\item UC-X Nel caso in cui l'utente tenti l'inserimento di campi non validi vedrà comparire messaggi d'errore.
				\end{itemize}
		\end{itemize}
		
	\subsubsection{UC-X Conferma richiesta insegnante}
		\begin{itemize}
			\item Attori: Amministratore;
			\item Precondizione: l'amministratore si trova nella vista principale dell'applicazione;
			\item Postcondizione: l'amministratore ha confermato l'utente richiedente il ruolo di insegnante.
			\item Scenario principale:
				\begin{enumerate}
					\item l'amministratore visualizza la lista degli utenti che richiedono il ruolo di insegnante;
					\item l'amministrazione selezione l'utente da confermare;
				\end{enumerate}
		\end{itemize}
					
	\subsubsection{UC-X Rifiuto richiesta insegnante}
		\begin{itemize}
			\item Attori: Amministratore;
			\item Precondizione: l'amministratore si trova nella vista principale dell'applicazione;
			\item Postcondizione: l'amministratore ha rifiutato l'utente richiedente il ruolo di insegnante.
			\item Scenario principale:
				\begin{enumerate}
					\item l'amministratore visualizza la lista degli utenti che richiedono il ruolo di insegnante;
					\item l'amministrazione selezione l'utente da rifiutare;
				\end{enumerate}
		\end{itemize}
					
					
					
\subsection{Attore Allievo}
	\subsubsection{UC-X Ricerca esercizi disponibili}
		\begin{itemize}
			\item Attori: allievo, insegnante;
			\item Precondizione: l'utente si trova nella vista principale dell'applicazione;
			\item Postcondizione: l'utente ottiene una lista degli esercizi filtrati.
			\item Scenario principale:
				\begin{enumerate}
					\item l'utente accede all'area dedicata alla ricerca degli esercizi;
					\item l'utente può filtrare l'esercizio in base al suo nome;
					\item l'utente può filtrare l'esercizio in base alla frase;
					\item l'utente può filtrare l'esercizio in base all'autore;
					\item l'utente può filtrare l'esercizio in base agli argomenti trattati;
				\end{enumerate}
			\item Estensioni:
				\begin{itemize}
					\item UC-X Nel caso in cui l'utente tenti l'inserimento di campi non validi vedrà comparire messaggi d'errore.
				\end{itemize}
		\end{itemize}

	\subsubsection{UC-X Esecuzione esercizio}
		\begin{itemize}
			\item Attori: Allievo
			\item Precondizione: l'allievo ha selezionato un esercizio da eseguire;
			\item Postcondizione: l'allievo ha terminato i quesiti proposti dell'esercizio.
			\item Scenario principale:
				\begin{enumerate}
					\item l'allievo sceglie la classe grammaticale per ogni parola presentata.
				\end{enumerate}
			\item Estensioni: 
				\begin{itemize}
					\item UC-X Nel caso in cui l'utente tenti l'inserimento di campi non validi vedrà comparire messaggi d'errore.
				\end{itemize}
			\item Inclusioni:
				\begin{itemize}
					\item UC-X Nel caso in cui l'esercizio sia terminato;
				\end{itemize}
			\end{itemize}

	\subsubsection{UC-X Valutazione esercizio}
	\begin{itemize}
			\item Attori: Allievo
			\item Precondizione: L'allievo ha completato l'esecuzione dell'esercizio;
			\item Postcondizione: L'allievo visualizza la valutazione dell'esercizio.
			\item Scenario principale:
				\begin{enumerate}
					\item l'allievo al termine dell'esecuzione dell'esercizio riceve la valutazione.
				\end{enumerate}
			\end{itemize}
			
	\subsubsection{UC-X Visualizzazione progressi}
	\begin{itemize}
			\item Attori: Allievo
			\item Precondizione: L'allievo si trova nella vista principale dell'applicazione;
			\item Postcondizione: L'allievo visualizza i progressi svolti fino a quel momento.
			\item Scenario principale:
				\begin{enumerate}
					\item l'allievo prima e/o dopo l'esecuzione di uno o più esercizi può visualizzare i progressi raggiunti.
				\end{enumerate}
			\end{itemize}

La nostra architettura utilizza il framework$^{*}$ Express e del servizio di database fornito da Firebase (Firestore). L'architettura dell'applicazione è client-server$^{*}$ ed è costituita da tre principali parti: model (che gestisce le operazioni sul database), presenter (che gestisce la logica dell'applicazione) e view (che gestisce la visualizzazione delle pagine web). Il sistema permette di:
\begin{itemize}
	\item lo svolgimento di esercizi di analisi grammaticale (e loro valutazione) e la scrittura dei dati, come ad esempio la soluzione, su un database Firestore messo a disposizione da Firebase;
	\item la visualizzazione e l'utilizzo di un software di apprendimento automatico per il riconoscimento delle classi grammaticali delle parole di una frase.
\end{itemize}

\begin{figure}[h]
	\includegraphics[scale=0.4]{images/architettura.png}
	\caption{Schema generale dell'applicazione}
\end{figure}

Più in dettaglio, abbiamo suddiviso ulteriormente il model in:
\begin{itemize}
	\item POSManager: parte dell'applicazione che si occupa dell'uso del software di apprendimento automatico;
	\item Data: l'insieme delle classi di business che permettono di svolgere le operazione di calcolo sui dati estratti dal database;
	\item Firebase: insieme di classi usate per l'utilizzo del database fornito da Firebase;
	\item Database: insieme di classi che utilizzano quelle definite dal pacchetto Firebase e che disaccoppiano l'applicazione dall'implementazione del database;
	\item Client: il cui scopo è esporre le funzionalità del model.
\end{itemize}
\begin{figure}[h]
	\includegraphics[scale=0.5]{images/package.png}
	\caption{Diagramma dei package}
\end{figure}


\newpage
\subsection{View}
\includegraphics[scale=0.41]{images/View.png}
\newpage
\subsection{Presenter}
I presenter sono organizzata specularmente alle viste ed espongono le funzionalità necessarie per le pagine che gestiscono e gestiscono la logica di controllo prelevando le informazioni dal modello.
\begin{itemize}
	\item \texttt{DeveloperPresenter}:  
	\item \texttt{ProfilePresenter}: 
	\item \texttt{SearchPresenter}: 
	\item \texttt{ExercisePresenter}: 
\end{itemize}

\includegraphics[scale=0.53]{images/Presenter.png}


\newpage
\subsection{Model}
\subsubsection{Data}
\begin{figure}[ht]
	\includegraphics[scale=0.45]{images/Data.png}
	\caption{Diagramma delle classi del package Data}
\end{figure}

In Data si trovano tutte le classi di business dell'applicazione che permettono la risoluzione degli esercizi e di calcolare valutazioni di esercizi, medie degli utenti. Le classi rispecchiano la rappresentazione della base informativa.

\begin{itemize}
	\item \texttt{Data}: interfaccia che identifica tutti i dati rappresentati nel database;
	\item \texttt{Exercise}: modellazione di un esercizio presente nel database che espone le funzionalità necessarie allo svolgimento e alla valutazione di un esercizio da parte di un particolare insegnante;
	\item	\texttt{Solution}: modellazione di una soluzione di un esercizio che permette la valutazione della soluzione;
	\item \texttt{Class}: modellazione delle classi contenute nella piattaforma che mette in relazione studenti e insegnanti;
	\item \texttt{User}: modellazione di un utente qualsiasi registrato nella piattaforma che permette il calcolo delle informazioni correlate a tali utenti:
	\begin{itemize}
		\item \texttt{Teacher}: modellazione di un insegnante che permette il calcolo delle classi di cui esso è insegnante;
		\item \texttt{Student}: modellazione di uno studente che permette il calcolo dell'andamento della media delle valutazioni e delle classi a cui appartiene.
	\end{itemize}
\end{itemize}


\subsubsection{POSManager}
POSManager fornisce il modo di svolgere esercizi tramite il software di POS-tagging. Il pacchetto fornisce un'interfaccia che ogni software di questo tipo deve soddisfare, ovvero le operazioni di training e tagging e la capacità di restituire il risultato di quest'ultima operazione. In particolare, HunposManager è dotato di ulteriori funzioni che permettono la costruzione della soluzione. 

\begin{figure}[ht]
	\centering
	\includegraphics[scale=1]{images/POSManager.png}
	\caption{Diagramma delle classi del package POSManager}
\end{figure}
\newpage

\subsubsection{Database}
Database gestisce l'utilizzo del database tramite le classi DatabaseManager e derivate. La suddivisione rispecchia la rappresentazione dei dati all'interno della base di dati. In particolare:
\begin{itemize}
	\item \texttt{DatabaseExerciseManager}: permette le operazioni CRUD$^{*}$ e di ricerca degli esercizi;
	\item \texttt{DatabaseClassManager}: permette le operazioni CRUD e di ricerca delle classi;
	\item \texttt{DatabaseUserManager}: permette le operazioni CRUD e di ricerca degli User.
\end{itemize}

\begin{figure}[h]
	\includegraphics[scale=0.5]{images/DatabaseManager.png}
	\caption{Diagramma delle classi del package Database}
\end{figure}

\newpage
\subsubsection{Firebase}
Firebase rappresenta l'implementazione del database dell'applicazione e permette l'uso del database Firestore fornito da Firebase. In particolare:
\begin{itemize}
	\item \texttt{FirebaseExerciseManager}: permette le operazioni CRUD e di ricerca degli esercizi su Firestore;
	\item \texttt{FirebaseClassManager}: permette le operazioni CRUD e di ricerca delle classi su Firestore;
	\item \texttt{FirebaseUserManager}: permette le operazioni CRUD e di ricerca degli User su Firestore.
\end{itemize}

\begin{figure}[h]
	\includegraphics[scale=0.5]{images/FirebaseManager.png}
	\caption{Diagramma delle classi del package Firebase}
\end{figure}


Questa gerarchia è stata organizzata come un singleton con supporto per il subclassing. Ogni sottoclasse di \texttt{FirebaseManager} registrerà la propria istanza in \texttt{FirebaseManager.registry}. Tale istanza sarà recuperabile tramite il metodo \texttt{getInstance}. In questo modo, il metodo \texttt{Firebase.initDB}, che deve essere chiamato una sola volta all'interno dell'applicazione, verrà invocato solo alla prima istanza di una sottoclasse di \texttt{FirebaseManager}.

\newpage
\subsubsection{Client}
Client è composto da diverse classi che espongono le varie funzionalità sul modello. In particolare, la classe \texttt{Client} è una composizione di funzionalità fornite da:
\begin{itemize}
	\item \texttt{ExerciseClient}: fornisce le funzionalità di inserimento, risoluzione e ricerca degli esercizi;
	\item \texttt{ClassClient}: fornisce le funzionalità di inserimento e recupero delle informazioni delle classi e l'assegnazione degli esercizi;
	\item \texttt{UserClient}: fornisce le funzionalità di inserimento e verifica dell'identità degli utenti.
\end{itemize}

\begin{figure}[ht]
	\includegraphics[scale=0.5]{images/Client.png}
	\caption{Diagramma delle classi del package Client}
\end{figure}


\newpage
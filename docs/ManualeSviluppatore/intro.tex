\subsection{Scopo del documento}
L'obiettivo di questo documento è quello di fornire una guida introduttiva per la web application Colletta. Questo documento è  diretto a tutti gli sviluppatori interessati a contribuire al miglioramento del software realizzato dal gruppo OttoBit. Verranno illustrate le attività necessarie alla corretta installazione del prodotto e alla configurazione dell'ambiente di sviluppo. Verranno inoltre esposte in dettaglio le tecnologie interessate e l'architettura del progetto e come agire su di essa per estenderne le funzionalità.
\subsection{Scopo del prodotto}
Lo scopo del prodotto è creare una piattaforma collaborativa di raccolta dati su cui sia possibile predisporre e/o svolgere esercizi di analisi grammaticale. La raccolta vorrebbe avere il fine di fornire a sviluppatori e ricercatori dati sufficienti per applicare metodi di apprendimento automatico$^*$. Nello specifico si vorrebbe poter insegnare ad un elaboratore a svolgere gli stessi esercizi proposti agli utenti, divenendo una sorta di correttore automatico.  
Le componenti principali del prodotto saranno quindi:
\begin{itemize}
	\item un'interfaccia web, su cui verranno predisposti e svolti gli esercizi;
	\item un servizio esistente di database$^*$;
	\item il servizio esistente open-source per il pos-tagging.
\end{itemize}
\subsection{Riferimenti}
Segue l'elenco dei riferimenti utilizzati in questo documento:
\subsubsection{Di installazione}
	\begin{itemize}
	\item Download applicazione Colletta
	\footnote{\url {https://github.com/ottoBitPd/colletta}}
	\item Download Node.js
	\footnote{\url {https://nodejs.org/en/download/}}
	\item Download Webstorm
	\footnote{\url {https://www.jetbrains.com/webstorm/download}}
	\item Download TSLint
	\footnote{\url {https://www.npmjs.com/package/tslint}}
	\item Download JSDoc
	\footnote{\url {https://www.npmjs.com/package/jsdoc}}
	
\end{itemize}
\subsubsection{Informativi}
	\begin{itemize}
	\item Informazioni su Node.js
	\footnote{\url {https://nodejs.org/en/}}
	\item Informazioni su Webstorm
	\footnote{\url {https://www.jetbrains.com/webstorm/}}
	\item Informazioni su JSDoc
	\footnote{\url {https://www.npmjs.com/package/jsdoc}}
	\item Informazioni su Express.js
	\footnote{\url {https://expressjs.com/it/}}
	\item Informazioni su express-session
	\footnote{\url {https://www.npmjs.com/package/express-session}}
	\item Informazioni su shelljs
	\footnote{\url {https://www.npmjs.com/package/shelljs}}
	\item Informazioni su file-system
	\footnote{\url {https://www.npmjs.com/package/file-system}}
	\item Informazioni su bcryptjs
	\footnote{\url {https://www.npmjs.com/package/bcryptjs}}
\end{itemize}
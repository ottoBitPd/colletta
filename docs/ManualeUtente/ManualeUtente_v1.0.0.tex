\documentclass[11pt,a4paper]{article}
\usepackage[utf8]{inputenc}
\usepackage[italian]{babel}
\usepackage{amsmath}
\usepackage{amsfonts}
\usepackage{amssymb}
\usepackage{array}
\usepackage{graphicx}
\usepackage{multirow}
\usepackage{color,colortbl}
\usepackage[hidelinks]{hyperref}
\usepackage{fancyhdr}
\usepackage{tabularx}
\usepackage[left=2cm,right=2cm,top=2cm,bottom=3cm]{geometry}
\usepackage{enumerate}
\usepackage{lastpage}
\usepackage{hyperref}
\usepackage{titlesec}
\usepackage{ltablex}
\hypersetup{colorlinks,urlcolor=blue,linkcolor=black}
\usepackage{listings}
\usepackage{multicol}
\setlength{\columnsep}{1cm}


\definecolor{lightgray}{rgb}{.9,.9,.9}
\definecolor{darkgray}{rgb}{.4,.4,.4}
\definecolor{purple}{rgb}{0.65, 0.12, 0.82}

\lstdefinelanguage{JavaScript}{
	keywords={typeof, new, true, false, catch, function, return, null, catch, switch, var, if, in, while, do, else, case, break},
	keywordstyle=\color{blue}\bfseries,
	ndkeywords={class, export, boolean, throw, implements, import, this},
	ndkeywordstyle=\color{darkgray}\bfseries,
	identifierstyle=\color{black},
	sensitive=false,
	comment=[l]{//},
	morecomment=[s]{/*}{*/},
	commentstyle=\color{purple}\ttfamily,
	stringstyle=\color{red}\ttfamily,
	morestring=[b]',
	morestring=[b]"
}

\lstset{
	language=JavaScript,
	backgroundcolor=\color{lightgray},
	extendedchars=true,
	basicstyle=\footnotesize\ttfamily,
	showstringspaces=false,
	showspaces=false,
	numbers=left,
	numberstyle=\footnotesize,
	numbersep=9pt,
	tabsize=2,
	breaklines=true,
	showtabs=false,
	captionpos=b
}

\renewcommand{\lstlistingname}{Esempio}% Listing -> Algorithm
\renewcommand{\lstlistlistingname}{Elenco degli Esempi}% List of Listings -> List of Algorithms

\pagestyle{fancy}
\fancyhf{}
\lhead{\includegraphics[scale=0.07]{images/logo.png}}

\renewcommand {\footrulewidth}{0.2mm}
\lfoot {Norme di progetto}
\rfoot{Pagina \thepage\ di \pageref{LastPage}}

\definecolor{LightBlue}{rgb}{0,0,0.5}
\definecolor{Gray}{gray}{0.8}
\definecolor{LightGray}{gray}{0.9}

\usepackage{lipsum}
\usepackage{verbatim}


\setcounter{tocdepth}{4}
\setcounter{secnumdepth}{4}

\titlespacing*{\subsection}{0pt}{8ex plus 1ex minus .2ex}{2.3ex plus .2ex}
\titlespacing*{\subsubsection}{0pt}{8ex plus 1ex minus .2ex}{2.3ex plus .2ex}

\begin{document}
	\begin{titlepage}
  \centering
	\scshape
	
	\vspace*{2cm}
	\includegraphics[scale=0.7]{images/logo.png}
	\rule{\linewidth}{0.2mm}\\[0.37cm]
	{\Huge Piano di Qualifica}\\
	\rule{\linewidth}{0.2mm}\\[1cm]
	{\LARGE\bfseries Progetto Colletta - Gruppo OttoBit}\\[1cm]
	
	
	
	\begin{tabular}{>{\columncolor{Gray}}r | >{\normalfont}l}
		\rowcolor{LightBlue}		
		\multicolumn{2}{c}{\color{white}{Studio di fattibilità}}\\
		Versione & 0.0.1 \\
		Redazione & Giovanni Peron\\
 		Verifica & Michele Bortone\\
 		Responsabile & Benedetto Cosentino\\
 		Uso & Interno\\
 																 		& Prof. Tullio Vardanega\\
 																		& Prof. Riccardo Cardin\\
 		\multirow[t]{-3}{*}{Destinatari}	& MIVOQ s.r.l\\
 		\hline
	\end{tabular}
\end{titlepage}
	
	{\def\arraystretch{2}\tabcolsep=10pt
	\newpage
	\section*{\centering Registro delle modifiche}
	\begin{tabularx}{\textwidth}{ c | c | p{3.80cm} | c | X }
		\rowcolor{LightBlue}
		\color{white}\bfseries Versione & \color{white}\bfseries Data & \multicolumn{1}{c}{\color{white}\bfseries Autore}
		& \color{white}\bfseries Ruolo & \multicolumn{1}{c}{\color{white}\bfseries Descrizione}\\[0.25cm]
		2.0.1 & 2019-03-18 & ... & Amministratore & ... \\ \hline
		2.0.0 & 2019-03-05 & ... & Responsabile & ... \\ \hline
		0.0.1 & 2019-03-05 & Enrico Marcato & Amministratore & Redazione struttura documento \\ 
	 \hline		
	\end{tabularx}
	
	\newpage	
	
	\renewcommand  \contentsname {\Large Indice} 
	
	\tableofcontents
	\newpage
	\listoffigures
	\lstlistoflistings
	\newpage
	
	\section{Introduzione}
	\subsection{Scopo del documento}
	Lo scopo di questo documento è quello di illustrare le funzionalità che l'applicazione possiede e spiegarne l'utilizzo. Il documento subirà delle modifiche fino a quando lo sviluppo dell'applicazione non sarà completato.
	\subsection{Scopo del prodotto}
	Lo scopo del prodotto è creare una piattaforma collaborativa di raccolta dati su cui sia possibile predisporre e/o svolgere esercizi di analisi grammaticale. La raccolta vorrebbe avere il fine di fornire a sviluppatori e ricercatori dati sufficienti per applicare metodi di apprendimento automatico$^*$. Nello specifico si vorrebbe poter insegnare ad un elaboratore a svolgere gli stessi esercizi proposti agli utenti, divenendo una sorta di correttore automatico.  
	Le componenti principali del prodotto saranno quindi:
	\begin{itemize}
		\item un'interfaccia web, su cui verranno predisposti e svolti gli esercizi;
		\item un servizio esistente di database$^*$;
		\item il servizio esistente open-source per il pos-tagging.
	\end{itemize}
	
	\subsection{Riferimenti}
	Segue l'elenco dei riferimenti utilizzati dal gruppo:
	\subsubsection{Riferimenti Normativi}

	\subsubsection{Riferimenti informativi}
	\begin{itemize}
		\item 
		
	\end{itemize}	
	\section{Download e installazione}
	\subsection{Requisiti}
	L'applicazione è supportata solo ed esclusivamente da dispositivi desktop.
	\smallskip
	Sistemi operativi preferibili:
	\begin{itemize}
		\item Windows 10;
		\item Ubuntu 18.04;
		\item MacOS 10.14.4.		
	\end{itemize}
	\smallskip
	Browser preferibili:
		\begin{itemize}
		\item Mozilla Firefox 66.0;
		\item Google Chrome 67.0;
		\item Safari 12.1 (solo per MacOS).		
	\end{itemize}
	\smallskip
	Framework richiesti:
		\begin{itemize}
		\item NodeJS \\ Scaricalo per il tuo sistema operativo da https://nodejs.org/en/download/ ;		
		\end{itemize}
	

	\subsection{Download}
	L'applicazione è disponibile al download al seguente indirizzo:
	https://github.com/ottoBitPd/colletta-poc
	
	\subsection{Installazione}
	Per eseguire l'applicazione seguire i seguenti passi:
	 \begin{enumerate}
	 	\item Installare le librerie necessarie.\\ Da terminale, posizionarsi sulla cartella principale e utilizzare il seguente comando:\begin{lstlisting}[numbers=none]
	 	npm install
	 	\end{lstlisting}
	 	\item Lanciare il server.\\ Sempre da terminale e nella cartella principale, utilizzare il seguente comando:\begin{lstlisting}[numbers=none]
	 	node index.js
	 	\end{lstlisting}
	 	\item Accedere all'applicazione.\\ Apri il tuo browser e vai all'indirizzo: http://localhost:8080
	 	
	 \end{enumerate}
	\section{Utilizzo}	
\newpage

	


\end{document}


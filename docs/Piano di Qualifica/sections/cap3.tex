\section{Gestione amministrativa della revisone}
gestione problemi in fase di verifica, tipo spiegare delle issue e del merge se ci sono conflitti cosa susccede se il documento viene respinto
\subsection{Comunicazione e risoluzione di anomalie}
parliamo di ticketing
\subsection{Procedure di controllo di qualità di processo}
Come già specificato per verificare la qualità dei processi adotteremo lo standard ISO/IEC 15504. Per far ciò il verificatore dovrà occuparsi di valutare i processi secondo i parameteri specificati dallo standard, consultabili nell'appendice A in fondo a questo documento.
Per ogni processo dovranno essere valutati nove attributi assegnando ad ognuno di essi uno dei quattro livelli di misura: 
\begin{itemize}
\item N not implemented, 
\item P partial implemented, 
\item L largely implemented, 
\item F fully implemented.
\end{itemize}
Infine basandosi sui risultati delle valutazioni di questi attributi verrà deciso il grado complessivo di maturazione del processo espresso con uno dei sei livelli offerti dallo standard. I risultati di queste verifiche dovranno essere documentati, per questo il verificatore si occuperà di compilare, con i dati raccolti, le tabelle per la valutazione dei processi previste nella quarta sezione di questo documento.
Durante tutta la fase di controllo della qualità del processo è consigliabile consultare il Piano di qualifica costantemente in particolare le appendici, questo per evitare imprecisioni nella valutazione della qualità sia essa dei processi o del prodotto.
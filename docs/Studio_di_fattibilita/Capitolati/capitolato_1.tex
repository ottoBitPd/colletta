	\section{Capitolato C1}
	\subsection{Descrizione generale}
	Il primo capitolato prevede la creazione di un sistema distribuito, per la gestione automatizzata e personalizzabile, da parte del destinatario, delle segnalazioni di diverse interfacce.\\ 
	Tutto questo realizzato attraverso un pattern di Publisher/Subscriber.
	\subsection{Finalità del progetto}
	Il progetto Butterfly deve essere in grado di interfacciarsi con diversi strumenti, recuperando ed intercettando le segnalazioni e provvedano a riportarle nella forma desiderata dall'utilizzatore finale. Deve accentrare e standardizzare queste segnalazioni per essere il più flessibile possibile.\\
	\'E composto da 4 componenti:
	\begin{itemize}
		\item \textbf{Producers:} hanno il compito di recuperare le segnalazioni e pubblicarle, sotto forma di messaggi, all’interno dei Topic adeguati.
		\item \textbf{Broker:} Strumento per istanziare e gestire i Topic.
		\item \textbf{Consumers:} I componenti consumer avranno il compito di abbonarsi ai Topic adeguati, recuperarne i	messaggi e procedere al loro invio verso i destinatari finali.
		\item \textbf{Componente custom specifico:} Per componente custom specifico si intende un componente mirato a risolvere un’esigenza specifica dell’azienda. 
	\end{itemize}
	\subsection{Tecnologia interessate}
	\begin{itemize}
		\item \textbf{Java:} sistema per lo sviluppo di software;
		\item \textbf{Python:} linguaggio di programmazione per sviluppare applicazioni distribuite;
		\item \textbf{Node.js:} piattaforma Open source event-driven per l'esecuzione di codice JavaScript, costruita sul motore JavaScript V8 di Google;
		\item \textbf{Apache Kafka:} piattaforma open source di stream processing, per la gestione di feed dati in tempo reale;		
		\item \textbf{Docker:} è una tecnologia di containerizzazione.
	\end{itemize}
	\subsection{Conclusione}
	
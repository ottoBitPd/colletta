\documentclass[11pt,a4paper]{article}
\usepackage[utf8]{inputenc}
\usepackage[italian]{babel}
\usepackage{amsmath}
\usepackage{amsfonts}
\usepackage{amssymb}
\usepackage{array}
\usepackage{graphicx}
\usepackage{multirow}
\usepackage{color,colortbl}
\usepackage[hidelinks]{hyperref}
\usepackage{fancyhdr}
\usepackage{tabularx}
\usepackage[left=2cm,right=2cm,top=2cm,bottom=2cm]{geometry}
\usepackage{hyperref}
\hypersetup{colorlinks,urlcolor=blue,linkcolor=black}

\pagestyle{fancy}
\lhead{\includegraphics[scale=0.07]{../images/logo.png}}

\definecolor{LightBlue}{rgb}{0,0,0.5}
\definecolor{Gray}{gray}{0.8}
\definecolor{LightGray}{gray}{0.9}

\usepackage{lipsum}
\begin{document}
	\begin{titlepage}
  \centering
	\scshape
	
	\vspace*{2cm}
	\includegraphics[scale=0.7]{../images/logo.png}
	\rule{\linewidth}{0.2mm}\\[0.37cm]
	{\Huge Verbale Esterno 2019-01-10}\\
	\rule{\linewidth}{0.2mm}\\[1cm]
	{\LARGE\bfseries Progetto Colletta - Gruppo OttoBit}\\[1cm]
	
	
	
	\begin{tabular}{>{\columncolor{Gray}}r | >{\normalfont}l}
		\rowcolor{LightBlue}		
		\multicolumn{2}{c}{\color{white}{Informazioni sul documento}}\\
		Redazione & Gianmarco Pettenuzzo\\
 		Responsabile & Benedetto Cosentino\\
 		Uso & Esterno\\
 																 		& Prof. Tullio Vardanega\\
 																		& Prof. Riccardo Cardin\\
 		\multirow[t]{-3}{*}{Destinatari}	& MIVOQ s.r.l\\
 		\hline
	\end{tabular}
\end{titlepage}

	\tableofcontents
	\newpage	
	
	\section{Informazioni sulla riunione}
	Durante la stesura dell’analisi dei requisiti, sono sorti una serie di dubbi che il gruppo OttoBit ha riassunto in una serie di domande da porre all'azienda proponente del capitolato d'appalto MIVOQ S.r.l, in particolare al responsabile Giulio Paci.
	Tali domande sono state effettuate via e-mail in data 2019-01-07.
	
	\section{Ordine del Giorno}
	Sono riportati i punti relativi all'ordine del giorno:
	\begin{itemize}
		\item Domande e risposte.
	\end{itemize}
	
	\section{Resoconto}
	\subsection{Domande e risposte}
	\begin{itemize}
		\item \textbf{L'applicazione richiesta è rivolta specificatamente alle scuole o ad un'utenza generica? In particolare, gli insegnanti devono essere dei veri e propri professori o potrebbero essere delle persone qualsiasi? Il dubbio è sorto riflettendo sulle conoscenze che un utente nel ruolo di insegnante dovrebbe avere e quindi delle garanzie che accompagnerebbero la possibile correzione da lui fornita per un esercizio.}\\
		Nell'idea originale l'applicazione è rivolta ad un'utenza generica (diciamo di apprendimento fra pari): la soluzione migliore sarebbe implementare un meccanismo per cui gli insegnanti validi siano selezionati naturalmente, un po' come avviene su stackoverflow e simili.
Si possono ovviamente prevedere casi d'uso in cui lo strumento possa essere utilizzato da insegnanti e allievi in ambito scolastico, ma lascio a voi la scelta.

	\item \textbf{Riteniamo che dovrebbe esserci, in ogni caso, una sorta di controllo su chi assumerà il ruolo di insegnante. Il gruppo propone di aggiungere il ruolo dell'amministratore, utente con maggiori permessi e con il dovere di verificare le credenziali di un altro utente prima che questi possa assumere il ruolo di insegnante.}\\
	La figura di un moderatore credo sia piuttosto utile. Più che impedire ad un utente di assumere il ruolo di insegnante, potrebbe certificarne il ruolo. In questo modo gli allievi potrebbero sapere se l'esercizio proviene da un'insegnante certificato o meno e scegliere autonomamente gli esercizi da svolgere.
	\item \textbf{L'amministratore sarebbe anche l'unica figura con la possibilità di eliminare dal sistema esercizi e/o utenti. Pensiamo che in alternativa sarebbe complesso far gestire l'eliminazione degli esercizi agli insegnanti o potrebbero rimanere degli esercizi errati all'interno del sistema.} \\
	Che restino esercizi errati nel sistema è inevitabile. Soprattutto per compiti come l'analisi grammaticale (o simili) per i quali spesso non si riesce a trovare un accordo o per questioni di ambiguità della frase o per divergenze di opinione.
La rimozione dovrebbe sempre essere consentita almeno a chi ha inserito l'esercizio. Consiglio anche di prevedere di mantenere più versioni di uno stesso esercizio e di legare l'esercizio all'insegnante che lo ha introdotto e corretto.
	\item \textbf{Tornando a parlare dell'insegnante, ritenete che sarebbe opportuno che avesse la possibilità di assegnare esercizi ad un gruppo specifico di allievi? Come se avesse delle classi con livelli di preparazione diversa a cui assegnare, ovviamente, esercizi differenti.} \\
	Nel caso d'uso per le scuole è fondamentale. In altri contesti è facoltativo. Per cui dipende da cosa decidete di fare a proposito di questo caso d'uso.
	\item \textbf{Gli allievi (intesi come qualcuno che vuole solamente svolgere un esercizio, ad esempio come ripasso prima di una verifica in classe) devono essere per forza registrati nel sistema? Anche se non hanno interesse a vedere i propri progressi nel tempo?} \\
	Non credo sia necessario. Ma non ho nulla in contrario nemmeno alla registrazione obbligatoria.
	\item \textbf{Vorremmo, infine, capire meglio la struttura dei modelli che dovrebbero poter creare gli sviluppatori, quali sono i parametri. Sarebbe possibile avere un esempio?} \\
	Per fornirvi un esempio concreto ho bisogno di un po' di tempo.
Diciamo che l'importante è che ci sia l'intera soluzione, inclusiva dell'input a cui fa riferimento.
Per capirci, se andate qui \url{http://mary.dfki.de:59125} e selezionate la voce istc-lucia e come output PARTOFSPEECH, potete vedere un esempio di un sistema semplificato di posta tagging. Quel sistema divide la frase di input in token e etichetta i token con le classi grammaticali (S sostantivo, V verbo, ...). Quel sistema è stato addestrato con uno strumento (opennlp) che nella fase di addestramento accetta una frase per riga, nella forma "token\_TAG". Per esempio:

Ciao\_I mondo\_S .\_FS

Altri sistemi, come hunpos, richiedono un token per riga e una riga vuota per ogni frase.		
	\end{itemize}
\end{document}
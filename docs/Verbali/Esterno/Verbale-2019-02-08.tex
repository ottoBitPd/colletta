\documentclass[11pt,a4paper]{article}
\usepackage[utf8]{inputenc}
\usepackage[italian]{babel}
\usepackage{amsmath}
\usepackage{amsfonts}
\usepackage{amssymb}
\usepackage{array}
\usepackage{graphicx}
\usepackage{multirow}
\usepackage{color,colortbl}
\usepackage[hidelinks]{hyperref}
\usepackage{fancyhdr}
\usepackage{tabularx}
\usepackage[left=2cm,right=2cm,top=2cm,bottom=2cm]{geometry}
\usepackage{hyperref}
\hypersetup{colorlinks,urlcolor=blue,linkcolor=black}

\pagestyle{fancy}
\lhead{\includegraphics[scale=0.07]{../images/logo.png}}

\definecolor{LightBlue}{rgb}{0,0,0.5}
\definecolor{Gray}{gray}{0.8}
\definecolor{LightGray}{gray}{0.9}

\usepackage{lipsum}
\begin{document}
	\begin{titlepage}
  \centering
	\scshape
	
	\vspace*{2cm}
	\includegraphics[scale=0.7]{../images/logo.png}
	\rule{\linewidth}{0.2mm}\\[0.37cm]
	{\Huge Verbale Esterno 2019-02-08}\\
	\rule{\linewidth}{0.2mm}\\[1cm]
	{\LARGE\bfseries Progetto Colletta - Gruppo OttoBit}\\[1cm]
	
	
	
	\begin{tabular}{>{\columncolor{Gray}}r | >{\normalfont}l}
		\rowcolor{LightBlue}		
		\multicolumn{2}{c}{\color{white}{Informazioni sul documento}}\\
		Redazione & Michele Bortone\\
 		Responsabile & Giovanni Peron\\
 		Uso & Esterno\\
 																 		& Prof. Tullio Vardanega\\
 																		& Prof. Riccardo Cardin\\
 		\multirow[t]{-3}{*}{Destinatari}	& MIVOQ s.r.l\\
 		\hline
	\end{tabular}
\end{titlepage}

	\tableofcontents
	\newpage	
	
	\section{Informazioni sulla riunione}
	Il team OttoBit ha richiesto un incontro con Giulio Paci, rappresentante dell'azienda proponente MIVOQ S.r.l, per chiarire alcuni dubbi riguardanti principalmente le tecnologie da impiegare nella piattaforma Colletta.
	\begin{itemize}
		\item Luogo: Sede MIVOQ s.r.l, Padova
		\item Data: 2019-02-08;
		\item Ora: 14.30-15.30;
		\item Partecipanti del team:
		\begin{itemize}
			\item Giovanni Bergo;
			\item Michele Bortone;
			\item Benedetto Cosentino;
			\item Enrico Marcato;
			\item Eleonora Peagno;
			\item Giovanni Peron;
			\item Gianmarco Pettenuzzo.
			
		\end{itemize}
	\item Partecipanti esterni:
	\begin{itemize}
		\item Giulio Paci.
	\end{itemize}
	\end{itemize}
	
	\section{Ordine del Giorno}
	Sono riportati i punti relativi all'ordine del giorno:
	\begin{itemize}
		\item Chiarimenti su software di Pos-Tagging;
		\item Sicurezza della piattaforma;
		\item Performance della piattaforma.
	\end{itemize}
	
	\section{Resoconto}
	\subsection{Software di Pos-Tagging}
	Il team discute insieme al rappresentante di MIVOQ s.r.l delle principali differenze tra i due software di Pos-Tagging da utilizzare proposti nel capitolato d'appalto, ovvero Hunpos e FreeLing.
	Inoltre vengono chiariti i concetti di "Modello" e di "Dataset" e le loro interazioni con i software sopracitati.
	\subsection{Sicurezza della piattaforma}
	Viene fatta chiarezza a proposito di un' eventuale gestione di input scorretti o volgari da parte di alcuni utenti. Si conclude ribadendo l'importanza di una figura di amministratore con il permesso di eliminare possibili esercizi ritenuti non consoni.
	\subsection{Performance della piattaforma}
	Discutendo con Giulio Paci è emerso che le performance di Colletta non sono un aspetto di vitale importanza e possono quindi essere lasciate in secondo piano.
	Per quanto riguarda il software di Pos-Tagging, ci si aspetta una correttezza delle soluzioni di circa il 95\%.
	\subsection{Dataset}
	Il rappresentante di MIVOQ s.r.l si è proposto di fornire al team OttoBit un dataset di partenza per allenare il software di Pos-Tagging. In questo modo, le soluzioni degli esercizi generate automaticamente dalla piattaforma potranno rispettare la percentuale di correttezza dichiarata precendentemente.
\end{document}
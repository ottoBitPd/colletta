\documentclass[11pt,a4paper]{article}
\usepackage[utf8]{inputenc}
\usepackage[italian]{babel}
\usepackage{amsmath}
\usepackage{amsfonts}
\usepackage{amssymb}
\usepackage{array}
\usepackage{graphicx}
\usepackage{multirow}
\usepackage{color,colortbl}
\usepackage[hidelinks]{hyperref}
\usepackage{fancyhdr}
\usepackage{tabularx}
\usepackage[left=2cm,right=2cm,top=2cm,bottom=2cm]{geometry}
\usepackage{hyperref}
\hypersetup{colorlinks,urlcolor=blue,linkcolor=black}

\pagestyle{fancy}
\lhead{\includegraphics[scale=0.07]{../images/logo.png}}

\definecolor{LightBlue}{rgb}{0,0,0.5}
\definecolor{Gray}{gray}{0.8}
\definecolor{LightGray}{gray}{0.9}

\usepackage{lipsum}
\begin{document}
	\begin{titlepage}
  \centering
	\scshape
	
	\vspace*{2cm}
	\includegraphics[scale=0.7]{../images/logo.png}
	\rule{\linewidth}{0.2mm}\\[0.37cm]
	{\Huge Verbale Esterno 2019-04-26}\\
	\rule{\linewidth}{0.2mm}\\[1cm]
	{\LARGE\bfseries Progetto Colletta - Gruppo OttoBit}\\[1cm]
	
	
	
	\begin{tabular}{>{\columncolor{Gray}}r | >{\normalfont}l}
		\rowcolor{LightBlue}		
		\multicolumn{2}{c}{\color{white}{Informazioni sul documento}}\\
		Redazione & Enrico Marcato\\
 		Responsabile & Eleonora Peagno\\
 		Uso & Esterno\\
 																 		& Prof. Tullio Vardanega\\
 																		& Prof. Riccardo Cardin\\
 		\multirow[t]{-3}{*}{Destinatari}	& MIVOQ s.r.l\\
 		\hline
	\end{tabular}
\end{titlepage}
	\newpage	
	
	\section{Informazioni sulla riunione}
	Il team OttoBit ha richiesto un incontro con Giulio Paci, rappresentante dell'azienda proponente MIVOQ S.r.l, per presentare quanto implementato fino a quel momento e per discutere della correttezza di quanto progettato.
	\begin{itemize}
		\item Luogo: Sede MIVOQ s.r.l, Padova
		\item Data: 2019-04-26;
		\item Ora: 10.30-11.30;
		\item Partecipanti del team:
		\begin{itemize}
			\item Benedetto Cosentino;
			\item Enrico Marcato;
			\item Giovanni Peron;
			\item Gianmarco Pettenuzzo.
			
		\end{itemize}
	\item Partecipanti esterni:
	\begin{itemize}
		\item Giulio Paci.
	\end{itemize}
	\end{itemize}
	
	\section{Ordine del Giorno}
	Sono riportati i punti relativi all'ordine del giorno:
	\begin{enumerate}
		\item Dimostrazione funzionamento del prodotto software;
		\item Filtri riguardanti la ricerca di dati da parte dello sviluppatore;
		\item Eventuale utilizzo di modelli alternativi;
		\item Eventuale utilizzo di un server remoto.
	\end{enumerate}
	
	\section{Resoconto}
	\subsection{Dimostrazione funzionamento prodotto software}
	Il team presenta al rappresentante di MIVOQ s.r.l le funzionalità del prodotto implementate fino a quel momento.
	Su richiesta vengono spiegate alcune scelte effettuate riguardanti l'utilizzo dell'apprendimento automatico Hunpos.
	\subsection{Filtri riguardanti la ricerca di dati da parte dello sviluppatore}
	Viene fatta chiarezza riguardo l'interesse che lo sviluppatore ha nei dati presenti nella piattaforma. Hanno un'importanza rilevante il numero di soluzioni, corrispondenti ad un singolo esercizio, inserite dagli insegnanti. Inoltre è consigliabile che si tenga traccia degli esercizi svolti sia da utenti non autenticati sia da allievi. Questi dati attualmente non vengono salvati. Ogni dato non deve essere riconducibile a informazioni personali che potrebbero non rispettare la privacy della persona.
	
	\subsection{Utilizzo di modelli alternativi}
	La possibilità di inserire diversi modelli comporterebbe delle modifiche al prodotto troppo onerose. 
	
	\subsection{Utilizzo di un server remoto.}
	Non è richiesto che il prodotto si trovi in un server remoto, è possibile procedere con il completamento dell'implementazione con un server locale.
	
\end{document}
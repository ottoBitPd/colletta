\documentclass[11pt,a4paper]{article}
\usepackage[utf8]{inputenc}
\usepackage[italian]{babel}
\usepackage{amsmath}
\usepackage{amsfonts}
\usepackage{amssymb}
\usepackage{array}
\usepackage{graphicx}
\usepackage{multirow}
\usepackage{color,colortbl}
\usepackage[hidelinks]{hyperref}
\usepackage{fancyhdr}
\usepackage{tabularx}
\usepackage[left=2cm,right=2cm,top=2cm,bottom=2cm]{geometry}

\pagestyle{fancy}
\lhead{\includegraphics[scale=0.08]{../images/logo.png}}

\definecolor{LightBlue}{rgb}{0,0,0.5}
\definecolor{Gray}{gray}{0.8}
\definecolor{LightGray}{gray}{0.9}

\usepackage{lipsum}
\begin{document}

\begin{titlepage}
  \centering
	\scshape
	
	\vspace*{2cm}
	\includegraphics[scale=0.7]{../images/logo.png}
	\rule{\linewidth}{0.2mm}\\[0.37cm]
	{\Huge Verbale 2018-12-04}\\
	\rule{\linewidth}{0.2mm}\\[1cm]
	{\LARGE\bfseries Progetto Colletta - Gruppo OttoBit}\\[1cm]
	
	
	
	\begin{tabular}{>{\columncolor{Gray}}r | >{\normalfont}l}
		\rowcolor{LightBlue}		
		\multicolumn{2}{c}{\color{white}{Informazioni sul documento}}\\
		Redazione & Enrico Marcato\\
 		Responsabile & Benedetto Cosentino\\
 		Uso & Interno\\
 																 		& Prof. Tullio Vardanega\\
 																		& Prof. Riccardo Cardin\\
 		\multirow[t]{-3}{*}{Destinatari}	& MIVOQ s.r.l\\
 		\hline
	\end{tabular}
\end{titlepage}

	\tableofcontents
	\newpage
	\section{Informazioni sulla riunione}
	\begin{itemize}
	\item Luogo: Aula 1A150, Torre Archimede;
	\item Data: 2018-12-04;
	\item Ora: 14:30 - 17:00;
	\item Partecipanti del team: 
	\begin{itemize}
		\item Giovanni Bergo;
		\item Michele Bortone;
		\item Benedetto Cosentino;
		\item Enrico Marcato;
		\item Eleonora Peagno;
		\item Giovanni Peron;
		\item Gianmarco Pettenuzzo.
	\end{itemize}
	\item Segretario: Peagno Eleonora.
	\end{itemize}

	\section{Ordine del giorno}
	Sono riportati i punti relativi all'ordine del giorno:
	\begin{enumerate}
	\item Nome e logo del gruppo;
	\item Scelta capitolato;
	\item Strumentazione;
	\item Assegnazione ruoli;
	\item Accenni a norme redazionali.
	\end{enumerate}
	
	\section{Resoconto}	
	\subsection{Nome e logo del gruppo}
	\textbf{OttoBit} è l'appellativo concordato dal gruppo. L'idea nasce dalla denominazione provvisoria di Gruppo 8, i membri hanno ritenuto gradevole l'associazione del numero otto alla parola bit, riferita all'unità di misura binaria. Si decide di adottare un logo inerente al nome.
	\subsection{Studio di fattibilità}
	I membri del gruppo esprimono le proprie preferenze riguardo la scelta del capitolato. Vengono considerate attitudini personali e conoscenze tecnologiche dei membri. L'interessa che prevale riguarda il capitolato C2: Colletta, piattaforma raccolta dati di analisi di testo. I membri prima di prendere una decisione definitiva, si assegnano i capitolati per uno studio più approfondito, al capitolato C2 i membri assegnati sono due.
	\subsection{Strumentazione}	
	I membri propongono e discutono riguardo gli strumenti che permettano un'organizzazione efficace ed efficiente del lavoro,  basandosi sulla tipologia del lavoro stesso e sulle proprie conoscenze. 
	\begin{itemize}
	\item  \textbf{Gmail} per aprire e gestire un indirizzo email;
	\item  \textbf{Git} come sistema di versionamento con la piattaforma \textbf{Gitlab} che ne semplifica l'utilizzo tra i membri;
	\item  \textbf{Trello} per il tracciamento delle attività del team, suddivise tra da fare, in lavorazione e fatte;
	\item  \textbf{Latex} per la redazione dei documenti;
	\item \textbf{Slack} per una facile comunicazione grazie alla creazione di canali.
	\end{itemize}
	\subsection{Assegnazione ruoli} 
	
	\begin{itemize}
	\item Responsabile: Benedetto Cosentino;
	\item Amministratori:
		\begin{itemize}
		\item Giovanni Bergo;
		\item Enrico Marcato;
		\item Eleonora Peagno;
		\item Gianmarco Pettenuzzo.
		\end{itemize}
	\item Verificatori:
		\begin{itemize}
			\item Michele Bortone;
			\item Giovanni Peron.
		\end{itemize}
	\end{itemize}
	\subsection{Scelta delle prime norme di Progetto}	
	\subsubsection{Scelta delle prime norme redazionali}
	Si realizza un documento che funge da template di partenza, dal quale in futuro verrà stilato ogni documento necessario. In esso sono riportate le informazioni necessarie e la tabella del registro delle attività da completare. Le norme tipografiche saranno specificate nel documento Norme di progetto e nel caso non lo fossero, si fa riferimento alle norme di dafault imposte in Latex.

\end{document}
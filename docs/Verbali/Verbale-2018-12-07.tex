\documentclass[11pt,a4paper]{article}
\usepackage[utf8]{inputenc}
\usepackage[italian]{babel}
\usepackage{amsmath}
\usepackage{amsfonts}
\usepackage{amssymb}
\usepackage{array}
\usepackage{graphicx}
\usepackage{multirow}
\usepackage{color,colortbl}
\usepackage[hidelinks]{hyperref}
\usepackage{fancyhdr}
\usepackage{tabularx}
\usepackage[left=2cm,right=2cm,top=2cm,bottom=2cm]{geometry}

\pagestyle{fancy}
\lhead{\includegraphics[scale=0.08]{images/logo.png}}

\definecolor{LightBlue}{rgb}{0,0,0.5}
\definecolor{Gray}{gray}{0.8}
\definecolor{LightGray}{gray}{0.9}

\usepackage{lipsum}
\begin{document}
	\begin{titlepage}
  \centering
	\scshape
	
	\vspace*{2cm}
	\includegraphics[scale=0.7]{images/logo.png}
	\rule{\linewidth}{0.2mm}\\[0.37cm]
	{\Huge Piano di Qualifica}\\
	\rule{\linewidth}{0.2mm}\\[1cm]
	{\LARGE\bfseries Progetto Colletta - Gruppo OttoBit}\\[1cm]
	
	
	
	\begin{tabular}{>{\columncolor{Gray}}r | >{\normalfont}l}
		\rowcolor{LightBlue}		
		\multicolumn{2}{c}{\color{white}{Studio di fattibilità}}\\
		Versione & 0.0.1 \\
		Redazione & Giovanni Peron\\
 		Verifica & Michele Bortone\\
 		Responsabile & Benedetto Cosentino\\
 		Uso & Interno\\
 																 		& Prof. Tullio Vardanega\\
 																		& Prof. Riccardo Cardin\\
 		\multirow[t]{-3}{*}{Destinatari}	& MIVOQ s.r.l\\
 		\hline
	\end{tabular}
\end{titlepage}

	\tableofcontents
	\newpage
	\section*{\centering Registro delle modifiche}
	\begin{tabularx}{\textwidth}{ c | c | c | c | X }
		\rowcolor{LightBlue}
		\color{white}\bfseries Versione & \color{white}\bfseries Data & \color{white}\bfseries Autore & \color{white}\bfseries Ruolo & \multicolumn{1}{c}{\color{white}\bfseries Descrizione}\\[0.25cm]
		0.0.1 & 2018-12-10 & Enrico Marcato & Amministratore & Stesura verbale \\
	\end{tabularx}

	\newpage
	\section{Informazioni sulla riunione}
	\begin{itemize}
	\item Luogo: Aula 2BC45, Torre Archimede;
	\item Data: 2018-12-07;
	\item Ora: 13:30 - 17:00;
	\item Partecipanti del team: 
	\begin{itemize}
		\item Giovanni Bergo;
		\item Michele Bortone;
		\item Benedetto Cosentino;
		\item Enrico Marcato;
		\item Eleonora Peagno;
		\item Giovanni Peron;
		\item Gianmarco Pettenuzzi.
	\end{itemize}
	\item Segretario: Peagno Eleonora.
	\end{itemize}

	\section{Ordine del giorno}
	Sono riportati i punti relativi all'ordine del giorno:
	\begin{enumerate}
	\item Conclusioni studio di fattibilità;
	\item Ulteriore discussione sulle norme;
	\item Scelta del modello di sviluppo; 
	\end{enumerate}
	
	\section{Resoconto}	
	\subsection{Conclusioni studio di fattibilità}
	\begin{itemize}
	\item Il gruppo legge insieme lo studio di fattibilità del capitolato 1 redatto da Giovanni Bergo;
	\item Il gruppo discute le opinioni sul capitolato 1 e ne trae le conclusioni;
	\item Viene confermato un modesto interesse del gruppo per l’argomento e per le tecnologie che verrebbero utilizzate per lo svolgimento.
	\item Il gruppo legge lo studio di fattibilità del capitolato 2 redatto da Benedetto Cosentino e Enrico Marcato;
	\item Il gruppo discute le opinioni sul capitolato 2 e ne trae le conclusioni;
	\item Il capitolato suscita particolare interesse al gruppo che prende in considerazione la possibilità di cimentarsi nello sviluppo di tale progetto;
	\item Il gruppo decide di approfondire gli argomenti correlati.
	\item Il gruppo legge insieme lo studio di fattibilità del capitolato 3 redatto da Giovanni Peron;
	\item Il gruppo discute le opinioni sul capitolato 3 e ne trae le conclusioni;
	\item Il progetto è stato trovato abbastanza stimolante, soprattutto per le tecnologie di cui richiede l’utilizzo. Lo scopo, invece, non è stato ritenuto sufficientemente interessante per rendere questo capitolato la prima scelta del gruppo. 
	\item Il gruppo legge insieme lo studio di fattibilità del capitolato 4 redatto da Gianmarco Pettenuzzo;
	\item Il gruppo discute le opinioni sul capitolato 4 e ne trae le conclusioni;
	\item Il gruppo è inizialmente interessato al progetto ma, considerando le tecnologie da utilizzare, l’interesse scema, ritenendole troppo poco stimolanti e dispersive.
	\item Il gruppo legge insieme lo studio di fattibilità del capitolato 5 redatto da Eleonora Peagno;
	\item Il gruppo discute le opinioni sul capitolato 5 e ne trae le conclusioni;
	\item Il gruppo non ha trovato interesse nello sviluppo di tale progetto ritenendo che il lavoro richiesto fosse troppo incentrato sulla gamification piuttosto che sulla creazione di una tecnologia nuova, penalizzando la parte pratica del capitolato.  
	\item Il gruppo legge insieme lo studio di fattibilità del capitolato 6 redatto da Michele Bortone;
	\item Il gruppo discute le opinioni sul capitolato 6 e ne trae le conclusioni;
	\item Il gruppo ha ritenuto la presentazione poco chiara e con contenuti dispersivi ma, il problema principale, è stata la poca fiducia ispirata dal proponente stesso. 
	\item Il gruppo discute e sceglie di concentrarsi sullo sviluppo del secondo capitolato.
	\end{itemize}

	\subsection{Ulteriore discussione sulle norme}
	\begin{itemize}
	\item La struttura dei verbali viene resa più precisa;
	\item Accordo riguardante uno standard per la nomenclatura di file, branch e documenti, il nome dei branches di features sarà: ATT(codiceissue).
	\end{itemize}
	\subsection{Scelta del modello di sviluppo}	
	\begin{itemize}
	\item Il gruppo riflette sul tipo di modello di sviluppo da adottare;
	\item Viene scelto di ispirarsi ad un modello di sviluppo di tipo evolutivo.
	\end{itemize}


\end{document}